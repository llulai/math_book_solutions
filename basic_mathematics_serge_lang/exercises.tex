\documentclass{report}
\def\MLine#1{\par\hspace*{-\leftmargin}\parbox{\textwidth}{\[#1\]}}
\usepackage{amsmath}
\begin{document}

Book: Basic Mathematics
Author: Serge Lang

\tableofcontents{}

\chapter{Numbers}

\section{The Integers}
No exercises in this section.

\section{Rules for Addition}
Justify each step, using commutativity and associativity in proving the following identities.

\begin{enumerate}
    \item (a + b) + (c + d) = (a + d) + (b + c)
    \item (a + b) + (c + d) = (a + c) + (b + d)
    \item (a - b) + (c - d) = (a + c) + (- b - d)
    \item (a - b) + (c - d) = (a + c) - (b + d)
    \item (a - b) + (c - d) = (a + d) - (c - b)
    \item (a - b) + (c - d) = -(b + d) + (a + c)
    \item (a - b) + (c - d) = -(b + d) - (-a - c)
    \item ((x + y) + z) + w = (x + z) + (y + w)
    \item (x - y) - (z - w) = (x + w) - y + z
    \item (x - y) - (z - w) = (x - z) - (w - y)
    \item Show that - (a + b + c) = -a + (-b) + (-c).
    \item Show that -(a -b - c) = -a + b + c.
    \item Show that -(a - b) = b - a.
    \item[] Solve for x in the following equations.
    \item -2 + x = 4
    \item 2 - x = 5
    \item x - 3 = 7
    \item -x + 4 = 1
    \item 4 - x = 8
    \item -5 - x = -2
    \item -7 + x = -10
    \item -3 + x = 4
    \item Prove the \textbf{cancellation law for addition:}
	    \MLine{\textnormal{If } a + b = a + c, \textnormal{then } b = c}
    \item Prove: If a + b = a, then b = 0
\end{enumerate}

\section{Rules for Multiplication}

\begin{enumerate}
	\item Express each of the following expressions in the form $2^m3^na^rb^s$, where $m$, $n$, $r$, $s$ are positve integers.
		\begin{enumerate}
			\item $8a^2b^3(27a^4)(2^5ab)$
			\item $16b^3a^2(6ab^4)(ab)^3$
			\item $3^2(2ab)^3(16a^2b^5)(24b^2a)$
			\item $24a^3(1ab^2)^3(3ab)^2$
			\item $(3ab)^2(27a^3b)(16ab^5)$
			\item $32a^4b^5a^3b^2(6ab^3)^4$
		\end{enumerate}
	\item Prove:
		\MLine{(a + b)^3 = a^3 + 3a^2b + 3ab^2 + b^3}
		\MLine{(a - b)^3 = a^3 - 3a^2b + 3ab^2 - b^3}
	\item Obtain expansion for $(a + b)^4$ and $(a - b)^4$ similar to the expansions for $(a + b)^3$ and $(a - b)^3$ of the preceding exercise.
	\item[] Expand the following expressions as sums of powers of $x$ multiplied by integers.
	\item $(2 - 4x)^2$
	\item $(1 - 2x)^2$
	\item $(2x + 5)^2$
	\item $(x-1)^2$
	\item $(x + 1)(x-1)$
	\item $(2x + 1)(x + 5)$
	\item $(x^2 + 1)(x^2 - 1)$
	\item $(1 + x^3)(1 - x^3)$
	\item $(x^2 + 1)^2$
	\item $(x^2 - 1)^2$
	\item $(x^2 + 2)^2$
	\item $(x^2 - 2)^2$
	\item $(x^2 - 4)^2$
	\item $(x^3 - 4)(x^3 + 4)$
	\item $(2x^2 + 1)(2x^2 - 1)$
	\item $(-2 + 3x)(-2 - 3x)$
	\item $(x + 1)(2x + 5)(x-2)$
	\item $(2x + 1)(1 - x)(3x + 2)$
	\item $(3x -1)(2x + 1)(x + 4)$
	\item $(-1 -x)(-2 + x)(1 - 2x)$
	\item $(-4x + 1)(2 - x)(3 + x)$
	\item $(1 -  x)(1 + x)(2 - x)$
	\item $(x - 1)^2(3 - x)$
	\item $(1 - x)^2(2-x)$
	\item $(1 - 2x)^2(3 + 4x)$
	\item $(2x + 1)^2(2 - 3x)$
	\item The population of a city in 1910 was 50,000, and it doubles every 10 years. What will it be (a) in 1970 (b) in 1990 (c) in 2000?
	\item The population of a city in 1905 was 100,00 and it doubles every 25 years. What will it be after (a) 50 years (b) 100 years (c) 150 years?
	\item The population of a city was 200 thousand in 1915, and it triples every 50 years. What will be the population:
		\begin{enumerate}
			\item in the year 2215?
			\item in the year 2165?
		\end{enumerate}
	\item The population of a city was 25,000 in 1870, and it triples every 40 years. What will it be:
		\begin{enumerate}
			\item in 1990?
			\item in 2030?
		\end{enumerate}
\end{enumerate}

\section{Even and Odd Integers; Divisibility}
\begin{enumerate}
	\item Give the proofs for the cases of Theorem 1 which were not proved in the text.
	\item Prove: If $a$ is even and $b$ is any positive integer, then $ab$ is even.
	\item Prove: If $a$ is even, then $a^3$ is even.
	\item Prove: If $a$ is odd, then $a^3$ is odd.
	\item Prove: if $n$ is even, then $(-1)^n = -1$.
	\item Prove: if $n$ is odd, then $(-1)^n = -1$.
	\item Prove: if $m$, $n$ are odd, then the product $mn$ is odd.
	\item[] Find the largest power of 2 which divides the following integers.
	\item 16
	\item 24 
	\item 32
	\item 20
	\item 50
	\item 64
	\item 100
	\item 36
	\item[] Find the largest power of 3 which divides the following integers.
	\item 30 
	\item 27
	\item 63
	\item 99
	\item 60
	\item 50
	\item 42
	\item 25
	\item Let $a$, $b$ be integers. Define $a \equiv b (\textnormal{mod } 5)$, which we read
		``$a$ is \textbf{congruent to} $b$ \textbf{modulo} 5'', to mean that
		$a - b$ is divisible by 5. Prove: if $a \equiv b (\textnormal{mod } 5)$ and $x \equiv y (\textnormal{mod } 5)$, then
		\MLine{a + x \equiv b + y (\textnormal{mod } 5)}
		and
		\MLine{ax \equiv by (\textnormal{mod } 5)}
	\item Let $d$ be a positive integer. Let $a$, $b$ be integers. Define
		\MLine{a \equiv b (\textnormal{mod } d)}
		to mean that $a - b$ is divisible by $d$. Prove that if $a \equiv b (\textnormal{mod } d)$ and $x \equiv y (\textnormal{mod } d)$, then
		\MLine{a + x \equiv b + y (\textnormal{mod } d)}
		and
		\MLine{ax \equiv by (\textnormal{mod } d)}
	\item Assume that every positive integer can be written in one of the forms $3k$, $3k + 1$, $3k + 2$ for some integer $k$.
		Show that if the square of a positive integer is divisible by 3, then so is the integer.
\end{enumerate}

\section{Rational Numbers}
\begin{enumerate}
	\item Solve for $a$ in the following equations.
		\begin{enumerate}
			\item $2a = \dfrac{3}{4}$
			\item $\dfrac{3a}{5} = -7$
			\item $\dfrac{-5a}{2} = \dfrac{3}{8}$
		\end{enumerate}
	\item Solve for $x$ in the following equations.
		\begin{enumerate}
			\item $3x - 5 = 0$
			\item $-2x + 6 = 1$
			\item $-7x = 2$
		\end{enumerate}
	\item Put the following fractions in the lowest form.
		\begin{enumerate}
			\item $ \dfrac{10}{25} $
			\item $\dfrac{3}{9}$
			\item $\dfrac{30}{25}$
			\item $\dfrac{50}{15}$
			\item $\dfrac{45}{9}$
			\item $\dfrac{62}{4}$
			\item $\dfrac{23}{46}$
			\item $\dfrac{16}{40}$
		\end{enumerate}
\end{enumerate}

\end{document}



