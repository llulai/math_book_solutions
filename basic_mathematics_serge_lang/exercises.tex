\documentclass{report}
\def\MLine#1{\par\hspace*{-\leftmargin}\parbox{\textwidth}{\[#1\]}}
\begin{document}

Book: Basic Mathematics
Author: Serge Lang

\tableofcontents{}

\chapter{Numbers}

\section{The Integers}
No exercises in this section.

\section{Rules for Addition}
Justify each step, using commutativity and associativity in proving the following identities.

\begin{enumerate}
    \item (a + b) + (c + d) = (a + d) + (b + c)
    \item (a + b) + (c + d) = (a + c) + (b + d)
    \item (a - b) + (c - d) = (a + c) + (- b - d)
    \item (a - b) + (c - d) = (a + c) - (b + d)
    \item (a - b) + (c - d) = (a + d) - (c - b)
    \item (a - b) + (c - d) = -(b + d) + (a + c)
    \item (a - b) + (c - d) = -(b + d) - (-a - c)
    \item ((x + y) + z) + w = (x + z) + (y + w)
    \item (x - y) - (z - w) = (x + w) - y + z
    \item (x - y) - (z - w) = (x - z) - (w - y)
    \item Show that - (a + b + c) = -a + (-b) + (-c).
    \item Show that -(a -b - c) = -a + b + c.
    \item Show that -(a - b) = b - a.
    \item[] Solve for x in the following equations.
    \item -2 + x = 4
    \item 2 - x = 5
    \item x - 3 = 7
    \item -x + 4 = 1
    \item 4 - x = 8
    \item -5 - x = -2
    \item -7 + x = -10
    \item -3 + x = 4
    \item Prove the \textbf{cancellation law for addition:}
	    \MLine{\textnormal{If } a + b = a + c, \textnormal{then } b = c}
    \item Prove: If a + b = a, then b = 0
\end{enumerate}

\section{Rules for Multiplication}

\begin{enumerate}
	\item Express each of the following expressions in the form $2^n3^na^rb^s$, where $m$, $n$, $r$, $s$ are positve integers.
		\begin{enumerate}
			\item $8a^2b^3(27a^4)(2^5ab)$
			\item $16b^3a^2(6ab^4)(ab)^3$
			\item $3^2(2ab)^3(16a^2b^5)(24b^2a)$
			\item $24a^3(1ab^2)^3(3ab)^2$
			\item $(3ab)^2(27a^3b)(16ab^5)$
			\item $32a^4b^5a^3b^2(6ab^3)^4$
		\end{enumerate}
	\item Prove:
		\MLine{(a + b)^3 = a^3 + 3a^2b + 3ab^2 + b^3}
		\MLine{(a - b)^3 = a^3 - 3a^2b + 3ab^2 - b^3}
	\item Obtain expansion for $(a + b)^4$ and $(a - b)^4$ similar to the expansions for $(a + b)^3$ and $(a - b)^3$ of the preceding exercise.
	\item[] Expand the following expressions as sums of powers of $x$ multiplied by integers.
	\item $(2 - 4x)^2$
	\item $(1 - 2x)^2$
	\item $(2x + 5)^2$
	\item $(x-1)^2$
	\item $(x + 1)(x-1)$
	\item $(2x + 1)(x + 5)$
	\item $(x^2 + 1)(x^2 - 1)$
	\item $(1 + x^3)(1 - x^3)$
	\item $(x^2 + 1)^2$
	\item $(x^2 - 1)^2$
	\item $(x^2 + 2)^2$
	\item $(x^2 - 2)^2$
	\item $(x^2 - 4)^2$
	\item $(x^3 - 4)(x^3 + 4)$
	\item $(2x^2 + 1)(2x^2 - 1)$
	\item $(-2 + 3x)(-2 - 3x)$
	\item $(x + 1)(2x + 5)(x-2)$
	\item $(2x + 1)(1 - x)(3x + 2)$
	\item $(3x -1)(2x + 1)(x + 4)$
	\item $(-1 -x)(-2 + x)(1 - 2x)$
	\item $(-4x + 1)(2 - x)(3 + x)$
	\item $(1 -  x)(1 + x)(2 - x)$
	\item $(x - 1)^2(3 - x)$
	\item $(1 - x)^2(2-x)$
	\item $(1 - 2x)^2(3 + 4x)$
	\item $(2x + 1)^2(2 - 3x)$
	\item The population of a city in 1910 was 50,000, and it doubles every 10 years. What will it be (a) in 1970 (b) in 1990 (c) in 2000?
	\item The population of a city in 1905 was 100,00 and it doubles every 25 years. What will it be after (a) 50 years (b) 100 years (c) 150 years?
	\item The population of a city was 200 thousand in 1915, and it triples every 50 years. What will be the population:
		\begin{enumerate}
			\item in the year 2215?
			\item in the year 2165?
		\end{enumerate}
	\item The population of a city was 25,000 in 1870, and it triples every 40 years. What will it be:
		\begin{enumerate}
			\item in 1990?
			\item in 2030?
		\end{enumerate}
\end{enumerate}

\end{document}
